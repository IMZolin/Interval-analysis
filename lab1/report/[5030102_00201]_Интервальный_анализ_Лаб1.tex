\documentclass[a4paper,14pt]{article}
\usepackage[a4paper, mag=1000, left=2.5cm, right=1cm, top=2cm, bottom=2cm, headsep=0.7cm, footskip=1cm]{geometry}
\usepackage[utf8]{inputenc}
\usepackage[T2A]{fontenc}
\usepackage[english,russian]{babel}
\usepackage{indentfirst}
%\usepackage[dvipsnames]{xcolor}
\usepackage[colorlinks]{hyperref}
\usepackage{amsfonts} 
\usepackage{amsmath}
\usepackage{graphicx}
\usepackage{float}

\DeclareGraphicsExtensions{.png,.jpg}

\usepackage{fancyhdr}
\pagestyle{fancy}
\fancyhead[LE,RO]{\thepage}
\fancyfoot{}

\usepackage{listings}

\hypersetup{linkcolor=black}

\title{non-linear equations}
\author{Иван Золин}
\date{2023}
\thispagestyle{empty}
\begin{document}
	
	\begin{titlepage}
		\begin{center}
			\textsc{
				Санкт-Петербургский политехнический университет имени Петра Великого \\[5mm]
				Физико-механический институт\\[2mm]
				Высшая школа прикладной математики и физики            
			}   
			\vfill
			\textbf{\large
				Интервальный анализ\\
				Отчёт по лабораторной работе №1 \\[3mm]
			}                
		\end{center}
		
		\vfill
		\hfill
		\begin{minipage}{0.5\textwidth}
			Выполнил: \\[2mm]   
			Студент: Золин Иван \\
			Группа: 5030102/00201\\
		\end{minipage}
		
		\hfill
		\begin{minipage}{0.5\textwidth}
			Принял: \\[2mm]
			к. ф.-м. н., доцент \\   
			Баженов Александр Николаевич
		\end{minipage}
		
		\vfill
		\begin{center}
			Санкт-Петербург \\2023 г.
		\end{center}
	\end{titlepage}
	
	\tableofcontents
	\newpage
	
	\section{Постановка задачи}
	Пусть дана вещественная матрица (1.1)
	\begin{equation}\text{mid} A = 
		\begin{pmatrix}
			a_{11} & a_{12}\\ 
			a_{21} & a_{22}
		\end{pmatrix}
	\end{equation}
	и неотрицательное число
	\begin{equation}
		\Delta \in [ 0, \min\{a_{ij}, i,j = \overline{1,2}\}]
	\end{equation}
	Рассмотрим две матрицы радиусов:
	\begin{equation}
		\text{rad}A_1 = \begin{pmatrix}
			1 & 1\\ 
			1 & 1
		\end{pmatrix}, \text{rad}A_2 = \begin{pmatrix}
		1 & 0\\ 
		1 & 0
	\end{pmatrix}
	\end{equation}
	Построим интервальную матрицу следующего вида
	\begin{equation}
		A = 
		\begin{pmatrix}
			[a_{11} - \Delta \cdot A^{(1,1)}_i, a_{11} + \Delta \cdot A^{(1,1)}_i ]  & [a_{12} - \Delta \cdot A^{(1,2)}_i, a_{12} + \Delta \cdot A^{(1,2)}_i ]\\ 
			[a_{21} - \Delta \cdot A^{(2,1)}_i, a_{21} + \Delta \cdot A^{(2,1)}_i ] & [a_{22} - \Delta \cdot A^{(2,2)}_i, a_{22} + \Delta \cdot A^{(2,2)}_i ]
		\end{pmatrix}
	\end{equation}
	$i=\overline{1,2}$
	Необходимо найти $\min\{\Delta|0 \in \det A\}$.
	В целях конкретизации и возможности проверки решения будем использовать следующую матрицу
	\begin{equation}
		\text{mid} A = \begin{pmatrix}
			1.05 & 1\\ 
			0.95 & 1
		\end{pmatrix}
	\end{equation}
	\section{Теория}
	Укажем основные арифметические операции для интервалов:
	\begin{equation}
		[a, b] + [c, d] = [a+c, b+d]
	\end{equation}
	\begin{equation}
		[a, b] - [c, d] = [a-d, b-c]
	\end{equation}
	\begin{equation}
	    [a, b] \cdot [c, d] = [\min(ac, ad, bc, bd), \max(ac, ad, bc, bd)]
	\end{equation}
	\begin{equation}
	    \frac{[a,b]}{[c,d]}=[\min\Big(\frac{a}{c}, \frac{a}{c}, \frac{b}{c}, \frac{b}{d}\Big), \max\Big(\frac{a}{c}, \frac{a}{d}, \frac{b}{c}, \frac{b}{d}\Big)]
	\end{equation}
	\begin{equation}
	   \text{mid}[a,b] = \frac{1}{2}(a+b)
	\end{equation}
	\begin{equation}
		\text{wid}[a,b] = (b-a)
	\end{equation}
	\begin{equation}
		\text{rad}[a,b] = \frac{1}{2}(b-a)
	\end{equation}
	Пусть $\text{mid} A = \{a_{ij} \}_{i,j \in N}$ – точечная вещественная матрица середин, $rad A = \{r_{ij}\}_{i,j \in N}$ – точечная вещественная матрица радиусов. Операцией midrad назовем следующую функцию:
	\begin{equation}
		\text{midrad}(\text{mid}A, \text{rad}A) = \{[\text{mid}A_{ij} - \text{rad} A_{ij} ], [\text{mid} A_{ij} + \text{rad} A_{ij} ]\}_{i,j \in N}
	\end{equation}
	Результатом операции является интервальная матрица.
	\section{Реализация}
	\subsection{Описание}
	Данная лабораторная работа была выполнена с использованием языка
	программирования Python 3.10 в среде разработки PyCharm.
	Дополнительно был реализован класс Interval, описывающий интервальную арифметику для удобства написания кода. 
	
	Отчёт подготовлен с помощью языка LaTEX в редакторе TexStudio.
	\subsection{Описание алгоритма}
	\begin{enumerate}
		\item Проверка вхождения нуля в интервал $\det A$ при максимально допустимом значении.
		\item Если $0 \notin \det A$, то данная задача не имеет решения. В противном случае переходим к шагу 3.
		\item Если $\det A$ представляет собой симметричныq интервал, то минимальное значение $\Delta$ устанавливается равным 0, так как
		$0 = $mid$[a, b]$.
		\item Рассмотрим весь допустимый интервал возможных значений $\Delta$. С использованием метода половинного деления будем сужать этот интервал до тех пор, пока не достигнем заданной точности $\varepsilon = 10^{-14}$.
	\end{enumerate}
	\subsection{Ссылка на репозиторий}
	\url{https://github.com/IMZolin/Interval-analysis} \ - GitHub репозиторий
	
	\section{Результат}
	\subsection{Первый случай матрицы радиусов}
	В соответствии с указанным алгоритмом мы получаем начальное значение $\Delta = 0.95$. Затем мы применяем операцию midrad к матрицам mid$A$ и rad$A_1$, что приводит к получению интервальной матрицы mid$A_1$:
	\begin{equation}
		\text{mid} A_1 = \begin{pmatrix}
			[0.1,2] & [0.05,1.95]\\ 
			[0,1.9] & [0.05,1.95]
		\end{pmatrix}
	\end{equation}
	Проверим вхождение нуля в $\det A_1$:
	\begin{equation}
		\det A_1 = [-3.7,3.9]
	\end{equation}
	Отсюда видно, что $0 \in \det 1$, а также mid$A_1 \neq 0$ значит, переходим к пункту 4 описанного алгоритма.
	В результате получаем $\min \Delta \approx 0.025$. В таком случае $\det A_1 = [2.220 · 10^{-16}, 0.2]$. Левый конец $\det A_1$
	с точностью до машинного эпсилон равен нулю.
	\subsection{Второй случай матрицы радиусов}
	Теперь применим операцию midrad к матрицам mid$A_1$ и rad$A_2$ и получим:
	\begin{equation}
		\text{mid} A_2 = \begin{pmatrix}
			[0.1,2] & 1\\ 
			[0,1.9] & 1
		\end{pmatrix}
	\end{equation}
	Проверим вхождение нуля в $\det A_2$:
	\begin{equation}
		\det A_2 = [-1.8,2]
	\end{equation}
	Вновь видим, что $0 \in \det A_2$, а также mid $A_2 \neq 0$, значит, переходим к пункту 4 алгоритма.
	В результате получаем $\min \Delta \approx 0.05$. В таком случае $\det A_2 = [1.110 \cdot 10^{-16}, 0.2]$. Левый конец $\det A_2$ с
	точностью до машинного эпсилон равен нулю.
	
	\section{Выводы}	
	Данные матрицы $A_1$,$\;A_2$ являются неособенной при $\Delta < 0.05$ и $\Delta \le 0.025$. $\Delta_1 > \Delta_2$, так как в 1-й задаче меньше интервальных элементов (2 интервала), чем во воторой задаче (4 интервала). При вычислении определителя происходит больше арифметических операций, при этом интервалы сужаться не могут, и поэтому детерминант быстрее начинает содержать ноль.
	\newpage
	\addcontentsline{toc}{section}{Литература}
	
	\begin{thebibliography}{4}
		\bibitem{s:hist}
		Histogram. URL: \url{https://en.wikipedia.org/wiki/Histogram}
		\bibitem{b:probSectMath}
		Вероятностные разделы математики. Учебник для бакалавров технических направлений.//Под ред. Максимова Ю.Д. --- Спб.: «Иван Федоров», 2001. --- 592 c., илл.
		\bibitem{s:boxplot}
		Box plot. URL: \url{https://en.wikipedia.org/wiki/Box_plot}
		\bibitem{a:nonParamRegr}
		Анатольев, Станислав (2009) «Непараметрическая регрессия», Квантиль, №7, стр. 37-52.
	\end{thebibliography}
	
\end{document}